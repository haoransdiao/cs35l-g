\documentclass{article}
\title{ChromaShare: Advanced Photo Organization and Searching}
\author {
	Diao, Haoran\\ \texttt{diode@ucla.edu}\\ \texttt{UID:205280122}
	\and
	Soykin, Elissa\\ \texttt{elissasoykin@ucla.edu}\\ \texttt{UID:905415537}
	\and
	Anderson, Alexa\\ \texttt{ack909@ucla.edu}\\ \texttt{UID:505518981}
	\and
	Zamparini, Simone\\ \texttt{simozampae@ucla.edu}\\ \texttt{UID:405360580}
}
\begin{document}
\maketitle
\section{Project Idea Summary}
	Most photo album apps that exist today allow for only the most
	basic forms of organization. In addition, most automatic
	organization uses arcane criteria that the end-user can not understand,
	resulting in unpredictable behavior and a poor UX.

	\textbf{Solution: ChromaShare}, an app which allows users to organize
	and share their photos through \textit{tags}. and automatically sorts
	photos into tags based off of their \textit{color histogram} and other color
	composition information, or through metadata such as location, title,
	and time. In addition, color and metadata information can be used to
	recommend tags from others.
\section{Features}

	\begin{itemize}
		\item Allow users to upload files along with a title and textual
		notes attached
		\item Allow users to create \textit{tags} which can be attached
		to any photo. Each photo can have multiple tags
		\item Automatically acquire the color histogram of photos, and
		use that, along with metadata such as title, location, and date
		to sort into tags consisting of photos with similar color
		histograms and metadata
		\item Show users the color histogram, as well as a palette of
		the most common colors in a photo
		\item Allow users to create accounts with usernames and
		passwords, as well as make the user's tags and tagged photos
		private or public to other users
		\item make suggestions to the user of (public) tags with color
		composition and metadata similar to the user's photos and tags
		\item Allow users to search through photos and tags with
		metadata and color histogram information.
	\end{itemize}

\section{Architecture}
\subsection{Backend}
Our backend will be written with \texttt{python} because it is familiar to most
of our group. Most heavy duty image processing will be done with the
\texttt{OpenCV} library so as to avoid any performance issues arising from using
python as a backend. For storage of user data, tag information, and photos, we
will use \texttt{MongoDB}, which is a relational database that will make
organizing our data easier. We will also be serving the webapp page ourselves
with the \texttt{http.server} python built in library. Communications to the
front end will be done through a JSON API also implemented with http.server.

\subsection{Frontend}
A webapp will be written with \texttt{React}, with requests to the server done
with \texttt{Jquery}, which has built in JSON parsing functions.

\subsection{Chart}
\begin{verbatim}
[Server Written in Python           ]          [WebApp written in React ]
[Serving the webapp with http.server]          [                        ]
[->communicating with the webapp wi-] JSON API [                        ]
[th a json API                      ] -------> [Jquery communicates wit-]
[processing images with OpenCV      ]          [Server                  ]
[storing data in MongoDB            ]          [                        ]
\end{verbatim}

\section{Summary}
This app will let people save their photos to another location. It offers more
options for customization than the default photo apps on phones.
Instead of putting a photo into an album on their phone, users can decide to tag
a photo with as many relevant ideas as they want. This will let them easily access
memories of events and locations they have visited; in addition, they can look
for photos related to various themes.

\end{document}
